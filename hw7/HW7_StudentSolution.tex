\newcommand{\MRFStudSolAA}{
%%%%%%%%%%%%%%%%%%%%%%%%%%%%%%%%%%%%
%%
%%.   YOUR SOLUTION FOR PROBLEM 1.a) BELOW THIS COMMENT
%%
%%%%%%%%%%%%%%%%%%%%%%%%%%%%%%%%%%%%
$$5^4$$
}
\newcommand{\MRFStudSolAB}{
%%%%%%%%%%%%%%%%%%%%%%%%%%%%%%%%%%%%
%%
%%.   YOUR SOLUTION FOR PROBLEM 1.b) BELOW THIS COMMENT
%%
%%%%%%%%%%%%%%%%%%%%%%%%%%%%%%%%%%%%
No, because the MRF has cycles in it, e.g. it is not a tree. 
}
\newcommand{\MRFStudSolAC}{
%%%%%%%%%%%%%%%%%%%%%%%%%%%%%%%%%%%%
%%
%%.   YOUR SOLUTION FOR PROBLEM 1.c) BELOW THIS COMMENT
%%
%%%%%%%%%%%%%%%%%%%%%%%%%%%%%%%%%%%%
Let's start by noticing that 
\[\mathcal{R} = \{ \{1\}, \{2\}, \{3\}, \{4\}, \{1, 2\}, \{1, 3\}, \{3, 4\}, \{2, 4\} \}\]
Since the number of messages is the number of Langrangian multipliers and we have one multiplier for each of the constraints $\sum_{\mathbf{y}_p\\\mathbf{y}_r} b_p(\mathbf{y}_b) = b_r(\mathbf{y}_r)$.\\
Since $P(r \in \{ \{1, 2\}, \{1, 3\}, \{3, 4\}, \{2, 4\}) = \varnothing$ only the other values of $r$ will generate a constraint. \\
\[\mathbf{y}_r \in \{1, 2, 3, 4, 5\} \forall r \in  \{ \{1\}, \{2\}, \{3\}, \{4\} \}\]
Finally, note that there are two parents for each of $r$, that is $|P(r)| = 2$.\\
Then there are $4\times5\times2 = 40$ constraints and, therefore, 40 messages.
}


\newcommand{\MRFStudSolBA}{
%%%%%%%%%%%%%%%%%%%%%%%%%%%%%%%%%%%%
%%
%%.   YOUR SOLUTION FOR PROBLEM 2.a) BELOW THIS COMMENT
%%
%%%%%%%%%%%%%%%%%%%%%%%%%%%%%%%%%%%%
Lets first find the function of $b_r$ that we are trying to maximize.
\begin{align*}
	\begin{bmatrix}
		b_1(0)\\
		b_1(1)\\
		b_2(0)\\
		b_2(1)\\
		b_{1, 2}(0, 0)\\
		b_{1, 2}(0, 1)\\
		b_{1, 2}(1, 0)\\
		b_{1, 2}(1, 1)
	\end{bmatrix}^\intercal
	\begin{bmatrix}
		\theta_1(0)\\
		\theta_1(1)\\
		\theta_2(0)\\
		\theta_2(1)\\
		\theta_{1, 2}(0, 0)\\
		\theta_{1, 2}(0, 1)\\
		\theta_{1, 2}(1, 0)\\
		\theta_{1, 2}(1, 1)
	\end{bmatrix} &= 
	\begin{bmatrix}
		b_1(0)\\
		b_1(1)\\
		b_2(0)\\
		b_2(1)\\
		b_{1, 2}(0, 0)\\
		b_{1, 2}(0, 1)\\
		b_{1, 2}(1, 0)\\
		b_{1, 2}(1, 1)
	\end{bmatrix}^\intercal
	\begin{bmatrix}
		1\\
		2\\
		1\\
		2\\
		-1\\
		2\\
		-1\\
		-1
	\end{bmatrix}\\
	&= b_1(0) + 2b_1(1) + b_2(0) + 2b_2(1) - \\
		& \hspace{0.5cm} b_{1, 2}(0, 0) + 2b_{1, 2}(0, 1) - b_{1, 2}(1, 0) - b_{1, 2}(1, 1)\\
	& := A_b
\end{align*}
Then the inference task can be rewritten as  
\begin{align*}
\max_{b} A_b \hspace{0.5cm} \text{s.t.} 
\begin{cases}
	b_r(\mathbf{y}_r) \in \{0, 1\} &\forall r, \mathbf{y}_r \\
	\sum_{\mathbf{y}_r} b_r(\mathbf{y}_r) = 1 &\forall r\\
	\sum_{\mathbf{y}_p / \mathbf{y}_r}  b_p(\mathbf{y}_p) = b_r(\mathbf{y}_r) &\forall r, p \in P(r), \mathbf{y}_r
\end{cases}
\end{align*}
Now let's  write the constraints explicitly. 

\begin{align*}
\max_{b} A_b \hspace{0.5cm} \text{s.t.} 
\begin{cases}
	b_1(0) \in \{0, 1\}\\
	b_1(1) \in \{0, 1\}\\
	b_2(0) \in \{0, 1\}\\
	b_2(1) \in \{0, 1\}\\
	b_{1,2}(0, 0) \in \{0, 1\}\\
	b_{1,2}(0, 1) \in \{0, 1\}\\
	b_{1,2}(1, 0) \in \{0, 1\}\\
	b_{1,2}(1, 1) \in \{0, 1\}\\
	b_1(0) + b_1(1) = 1\\
	b_2(0) + b_2(1) = 1\\
	b_{1,2}(0, 0) + b_{1,2}(0, 1) + b_{1,2}(1, 0) + b_{1,2}(1, 1) = 1\\
	b_{1, 2}(0, 0) + b_{1, 2}(0, 1) = b_1(0)\\
	b_{1, 2}(1, 0) + b_{1, 2}(1, 1) = b_1(1)\\
	b_{1, 2}(0, 0) + b_{1, 2}(1, 0) = b_2(0)\\
	b_{1, 2}(0, 1) + b_{1, 2}(1, 1) = b_2(1)
\end{cases}
\end{align*}
}

\newcommand{\MRFStudSolBB}{
%%%%%%%%%%%%%%%%%%%%%%%%%%%%%%%%%%%%
%%
%%.   YOUR SOLUTION FOR PROBLEM 2.b) BELOW THIS COMMENT
%%
%%%%%%%%%%%%%%%%%%%%%%%%%%%%%%%%%%%%
Note that the new cost function is 
\[ b_1(0) + 2b_1(1) + b_2(0) + 2b_2(1) - b_{1, 2}(0, 0) + 2b_{1, 2}(0, 1) - b_{1, 2}(1, 0) - b_{1, 2}(1, 1) \]
By simply looking at the previous item we can see that the maximum value is achieved when letting $b_{1, 2}(0, 1) = 1$. Then the solution is
The solution is:
\begin{align*}
b_1(0) &= 1\\
b_1(1) &= 0\\
b_2(0) &= 0\\
b_2(1) &= 1\\
b_{1, 2}(0, 0) &= 0\\
b_{1, 2}(0, 1) &= 1\\
b_{1, 2}(1, 0) &= 0\\
b_{1, 2}(1, 1) &= 0
\end{align*}
And then the value of our cost function is:
\[1 + 2(0) + 0 + 2(1) - 0 + 2(1) - 0 - 0 = 1 + 2 + 2 = 5\]
}

\newcommand{\MRFStudSolBC}{
%%%%%%%%%%%%%%%%%%%%%%%%%%%%%%%%%%%%
%%
%%.   YOUR SOLUTION FOR PROBLEM 2.c) BELOW THIS COMMENT
%%
%%%%%%%%%%%%%%%%%%%%%%%%%%%%%%%%%%%%
Because it becomes very slow for large programs. The reason being that it still has to do exhaustive search on the individual restrictions' arguments. e.g. $b_r(\mathbf{y}_r)$. Which, in large programs, may still be too large. Either $|r|$ is too large for some $r$ or there are too many restrictions, $|\mathcal{R}|$, is too large.
}