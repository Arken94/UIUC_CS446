% !TEX root = HW11.tex

\newcommand{\Dstar}{\frac{p_{data}(x)}{p_{data}(x) + p_G(x)}}

\newcommand{\GANStudSolA}{
%%%%%%%%%%%%%%%%%%%%%%%%%%%%%%%%%%%%
%%
%%.   YOUR SOLUTION FOR PROBLEM A BELOW THIS COMMENT
%%
%%%%%%%%%%%%%%%%%%%%%%%%%%%%%%%%%%%%
\[
	V(G_\theta, D_w) = \mathbb{E}_{x \sim p_X} \left[ \log D_w(x) \right] +
		\mathbb{E}_{z \sim Z} \left[ 1 - \log D_w \left( G_\theta(z) \right) \right]
\]
}

\newcommand{\GANStudSolB}{
%%%%%%%%%%%%%%%%%%%%%%%%%%%%%%%%%%%%
%%
%%.   YOUR SOLUTION FOR PROBLEM A BELOW THIS COMMENT
%%
%%%%%%%%%%%%%%%%%%%%%%%%%%%%%%%%%%%%
\begin{align*}
V(G_\theta, D) &= \int_x p_{data}(x) \log \left( D(x) \right) dx + \mathbb{E}_{z \sim Z} \left[ 1 - \log D_w \left( G_\theta(z) \right) \right]\\
&= \int_x p_{data}(x) \log \left( D(x) \right) dx + \int_z p_Z (z) \log \left( 1 - D \left( G_\theta (z) \right)\right) dz\\
&= \int_x p_{data}(x) \log \left( D(x) \right) dx + \int_x p_{G}(x) \log \left(1 - D(x) \right) dx\\
&= \int_x p_{data}(x) \log \left( D(x) \right) + p_{G}(x) \log \left(1 - D(x) \right) dx
\end{align*}

Where
\[
	\int_z p_Z (z) \log \left( 1 - D \left( G_\theta (z) \right)\right) dz = \int_x p_{G}(x) \log \left(1 - D(x) \right) dx
\]
follows from applying LOTUS to $\mathbb{E}_{z \sim Z} \left[ 1 - \log D_w \left( G_\theta(z) \right) \right]$ with the change of variable $X = G_\theta(Z)$. 
}

\newcommand{\GANStudSolC}{
%%%%%%%%%%%%%%%%%%%%%%%%%%%%%%%%%%%%
%%
%%.   YOUR SOLUTION FOR PROBLEM A BELOW THIS COMMENT
%%
%%%%%%%%%%%%%%%%%%%%%%%%%%%%%%%%%%%%
Following the Euler-Lagrange statement that the stationary point $q$ of
\[ S(q) = \int_a^b L(t, q(t), \dot{q}(t))dt \]
is given by the solution of 
$$
\frac{\partial L(t, q, \dot{q})}{\partial q} - \frac{d}{dx} \frac{\partial L(t, q, \dot{q})}{\partial \dot{q}} = 0
$$

In our case, let 
\[
	L(x, D, \dot{D}) = p_{data}(x) \log \left( D(x) \right) + p_{G}(x) \log \left(1 - D(x) \right) 
\]
So finding the optimal discriminator is the same thing as finding the stationary point in our Euler-Lagrange equation.
Then
\begin{align*}
	\frac{\partial L(t, q, \dot{q})}{\partial q} - \frac{d}{dx} \frac{\partial L(t, q, \dot{q})}{\partial \dot{q}} &= \frac{\partial L(t, q, \dot{q})}{\partial q} - \frac{d}{dx}(0)\\
	&= \frac{p_{data}}{D^*(x)} - \frac{p_G(x)}{1 - D^*(x)} = 0\\
	&\rightarrow p_{data}(x) - D^*(x) \left[ p_G(x) + p_{data}(x) \right] = 0\\
	&\rightarrow D^*(x) = \Dstar
\end{align*}
}

\newcommand{\GANStudSolD}{
%%%%%%%%%%%%%%%%%%%%%%%%%%%%%%%%%%%%
%%
%%.   YOUR SOLUTION FOR PROBLEM A BELOW THIS COMMENT
%%
%%%%%%%%%%%%%%%%%%%%%%%%%%%%%%%%%%%%

Let's first look at what value of $V(G, D*)$ we get when $p_G = p_{data}$. Note that a consequence of this is $D*(x) = \frac{1}{2}$. Then

\begin{align*}
	V(G, D^*) &= \int_x p_{data}(x) \log \frac{1}{2} + p_{G}(x) \log \frac{1}{2} dx\\
	&= \log \frac{1}{2} \int_x p_{data}(x) + p_{G}(x) dx\\
	&= 2 \log \frac{1}{2}\\
	&= - \log 4
\end{align*}

Given the previous result for the optimal $D$:
\[ D^*(x) = \Dstar \]
We can rewrite $V(G, D)$ as 
\begin{align*}
V(G, D^*) &= \int_x p_{data}(x) \log \left( D(x) \right) + p_{G}(x) \log \left(1 - D(x) \right) dx\\
&= \int_x p_{data}(x) \log \left( D(x) \right) + p_{G}(x) \log \left(1 - D(x) \right) dx\\
&= \int_x p_{data}(x) \log \left( \Dstar \right) + p_{G}(x) \log \left(1 - \Dstar \right) dx\\
&= \int_x p_{data}(x) \log \left(
	\frac{p_{data}(x)}{p_{data}(x) + p_G(x)}
\right) + p_{G}(x) \log \left(
	\frac{p_G(x)}{p_{data}(x) + p_G(x)}
\right) dx\\
&= \int_x p_{data}(x) \log \left(
	\frac{1}{2} \cdot \frac{p_{data}(x)}{\frac{p_{data}(x) + p_G(x)}{2}}
\right) + p_{G}(x) \log \left(
	\frac{1}{2} \cdot \frac{p_g(x)}{\frac{p_{data}(x) + p_G(x)}{2}}
\right) dx\\
&= \int_x p_{data}(x) \left(\log \frac{1}{2} + \log \left(
	\frac{p_{data}(x)}{\frac{p_{data}(x) + p_G(x)}{2}}
\right)\right) + p_{G}(x) \left(\log \frac{1}{2}  + \log \left(
	\frac{p_G(x)}{\frac{p_{data}(x) + p_g(x)}{2}}
\right)\right) dx\\
&= \log \frac{1}{2} \int_x p_{data}(x) + p_{G}(x) dx + 
\int_x p_{data}(x) \log \left(
	\frac{p_{data}(x)}{\frac{p_{data}(x) + p_G(x)}{2}}
\right)\\
& \hspace{2cm} + \int_x p_{G}(x) \log \left(
	\frac{p_G(x)}{\frac{p_{data}(x) + p_G(x)}{2}}
\right)dx\\
&= -\log 4 + D_{KL}(p_{data} || M) + D_{KL}(p_G || M)\\
&\ge -\log 4
\end{align*}
The inequality follows from the non-negativity of $D_{KL}$ (proven in the previous HW). From this we can conclude that $ -\log 4 $ is the global minimum and, therefore, the optimal generator $G^*(x)$ generates data with distribution $p^*_G = p_{data}$
}

\newcommand{\GANStudSolE}{
%%%%%%%%%%%%%%%%%%%%%%%%%%%%%%%%%%%%
%%
%%.   YOUR SOLUTION FOR PROBLEM A BELOW THIS COMMENT
%%
%%%%%%%%%%%%%%%%%%%%%%%%%%%%%%%%%%%%

KL divergence:\\
\[D_{KL}(\mathbb{P}_1 || \mathbb{P}_2) = \infty \]
Since $\mathbb{P}_2$ becomes 0 in a region of $\mathbb{P}_1$ domain and makes the $\log$ blow up to $\infty$. For the same reason:

\[D_{KL}(\mathbb{P}_1 || \mathbb{P}_3) = \infty \]\\


Wasserstein distances:\\
\[ \mathbb{W}_1(U, V) = \int_{\gamma \in \Gamma(U, V)} | F(U) - F(V) | d\gamma \] 
where $F$ denotes the CDF of the distribution. 

Then:
\begin{align*}
\mathbb{W}_1(\mathbb{P}_1, \mathbb{P}_2) &= 
	\int_{\gamma \in \Gamma(\mathbb{P}_1, \mathbb{P}_2)} | F(\mathbb{P}_1) - F(\mathbb{P}_2) | d\gamma\\
&= \int_{\gamma \in \Gamma(\mathbb{P}_1, \mathbb{P}_2)} | F(\mathbb{P}_1) - F(\mathbb{P}_2) | d\gamma\\
&= \int_0^{0.5} F(\mathbb{P}_1)d\gamma + \int_{0.5}^1 | F(\mathbb{P}_1) - F(\mathbb{P}_2) | d\gamma + \int_1^{1.5} F(\mathbb{P}_2)d\gamma\\
&= \int_0^{0.5} \frac{\gamma - 0}{1 - 0}d\gamma + 
	 \int_{0.5}^1 |  \frac{\gamma - 0}{1 - 0} -  \frac{\gamma - 0.5}{1.5 - 0.5} | d\gamma +
	 \int_1^{1.5} \frac{\gamma - 0.5}{1.5 - 0.5} d\gamma\\
&= \frac{\gamma^2}{2}\bigg\rvert_0^{0.5} + 
	0.5\gamma\bigg\rvert_{0.5}^1 +
	\left( 1.5\gamma - \frac{\gamma^2}{2}\right) \bigg\rvert_1^{1.5} \\
&= 0.5
\end{align*}

Following the same process (not shown):
\[
	\mathbb{W}_1(\mathbb{P}_1, \mathbb{P}_3) = 1
\]
}
 %The students have to fill this file to print the solution

% Problem Explanation:
% - first argument is the number of points
% - second argument is the title and the text
\examproblem{8}{Generative Adversarial Network (GAN)
}

%%%%%%%%%%%%%%%%%%%%%%%%%%%%%%%%%%%%%%
%%%%%  BEGINNING OF SUBPROBLEMS LIST
 \begin{enumerate}

 % Subproblem description
\examproblempart{What is the cost function for classical GANs? Use $D_w(x)$ as the discriminator and  $G_{\theta}(x)$ as the generator, where the generator transforms $z \sim Z$ to $x \in X$.\\}

\bookletskip{0.0}   %in inches

% Solution box
 \framebox[14.7cm][l]{
 \begin{minipage}[b]{14.7cm}
 \inbooklet{Your answer: \GANStudSolA}

 \end{minipage}
 }

 % Subproblem description
 \examproblempart{Assume arbitrary capacity for both discriminator and generator. In this case we refer to the discriminator using $D(x)$, and denote the distribution on the data domain induced by the generator via $p_G(x)$. State an equivalent problem to the one asked for in part (a), by using $p_G(x)$ and the ground truth data distribution $p_{data}(x)$.\\}

\bookletskip{0.0}   %in inches

% Solution box
 \framebox[14.7cm][l]{
 \begin{minipage}[b]{14.7cm}
 \inbooklet{Your answer: \GANStudSolB}

 \end{minipage}
 }

 \bookletpage
 % Subproblem description
 \examproblempart{Assuming arbitrary capacity, derive the optimal discriminator $D^*(x)$ in terms of $p_{data}(x)$ and $p_G(x)$.\\
 You may need the Euler-Lagrange equation:
$$
\frac{\partial L(x, D, \dot{D})}{\partial D} - \frac{d}{dx} \frac{\partial L(x, D, \dot{D})}{\partial \dot{D}} = 0
$$
where $\dot{D} = \partial D/\partial x$.

 }

\bookletskip{0.0}   %in inches

% Solution box
 \framebox[14.7cm][l]{
 \begin{minipage}[b]{14.7cm}
 \inbooklet{Your answer: \GANStudSolC}

 \end{minipage}
 }

% Subproblem description
\examproblempart{Assume arbitrary capacity  and an optimal discriminator $D^*(x)$, show that the optimal generator, $G^*(x)$, generates the distribution $p_G^* = p_{data}$, where
$p_{data}(x)$ is the data distribution\\
You may need the Jensen-Shannon divergence:
 $$
 \operatorname{JSD}(p_{\text{data}}, p_G) = \frac{1}{2} D_{KL}(p_{\text{data}}, M) + \frac{1}{2} D_{KL}(p_{G}, M) \quad\text{with}\quad M = \frac{1}{2}(p_{\text{data}} + p_G)
 $$
 }

\bookletskip{0.0}   %in inches

% Solution box
 \framebox[14.7cm][l]{
 \begin{minipage}[b]{14.7cm}
 \inbooklet{Your answer: \GANStudSolD}

 \end{minipage}
 }

% Subproblem description
\examproblempart{More recently, researchers have proposed to use the Wasserstein distance instead of divergences to train the models since the KL divergence often fails to give meaningful information for training. Consider three distributions, $\bP_1 \sim U[0,1]$, $\bP_2 \sim U[0.5,1.5]$, and $\bP_3 \sim U[1,2]$. Calculate $D_{KL}(\bP_1,\bP_2)$, $D_{KL}(\bP_1,\bP_3)$, $\bW_1(\bP_1, \bP_2)$, and $\bW_1(\bP_1, \bP_3)$, where $\bW_1$ is the Wasserstein-1 distance between distributions.\\}

\bookletskip{0.0}   %in inches

% Solution box
 \framebox[14.7cm][l]{
 \begin{minipage}[b]{14.7cm}
 \inbooklet{Your answer: \GANStudSolE}

 \end{minipage}
 }

 %%%%%%%%%%%% END OF SUBPROBLEMS LIST

\end{enumerate}
