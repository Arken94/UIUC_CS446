\newcommand{\HWStudSolAA}{
%%%%%%%%%%%%%%%%%%%%%%%%%%%%%%%%%%%%
%%
%%.   YOUR SOLUTION FOR PROBLEM 1.a) BELOW THIS COMMENT
%%
%%%%%%%%%%%%%%%%%%%%%%%%%%%%%%%%%%%%
Each $x^{(i)}$ is a set of features extracted from a given image. 
These features can be as simple as the RGB values for each of the pixels but can also be things more complex as edges (detected in the image). 
\vspace{2cm}
}
\newcommand{\HWStudSolAB}{
%%%%%%%%%%%%%%%%%%%%%%%%%%%%%%%%%%%%
%%
%%.   YOUR SOLUTION FOR PROBLEM 1.b) BELOW THIS COMMENT
%%
%%%%%%%%%%%%%%%%%%%%%%%%%%%%%%%%%%%%
It is each of the possible values in the "set of objects". If $S$ is the set of objects mentioned in the description then $y^{(i)} \in S$. For this particular setup $S$ could be something like \\ \{ cat, dog, tomato, potato, chair \} \\ in which case $y^{(i)}$ can be either one of cat, dog, tomato, potato or chair. 
\vspace{2cm}
}
\newcommand{\HWStudSolAC}{
%%%%%%%%%%%%%%%%%%%%%%%%%%%%%%%%%%%%
%%
%%.   YOUR SOLUTION FOR PROBLEM 1.c) BELOW THIS COMMENT
%%
%%%%%%%%%%%%%%%%%%%%%%%%%%%%%%%%%%%%
It is the convolutional neural network and its parameters that allow us to map the images ($x^{(i)}$) to a label ($y^{(i)}$).
\vspace{2cm}
}

\newcommand{\HWStudSolAD}{
%%%%%%%%%%%%%%%%%%%%%%%%%%%%%%%%%%%%
%%
%%.   YOUR SOLUTION FOR PROBLEM 1.d) BELOW THIS COMMENT
%%
%%%%%%%%%%%%%%%%%%%%%%%%%%%%%%%%%%%%
Learning is the process of tuning/fitting the parameters of the model in order to improve its predictions. Inference is the process of mapping a single instance of the inputs ($x$) to its possible label ($y$).
\vspace{2cm}
}

\newcommand{\HWStudSolBA}{
%%%%%%%%%%%%%%%%%%%%%%%%%%%%%%%%%%%%
%%
%%.   YOUR SOLUTION FOR PROBLEM 2.a) BELOW THIS COMMENT
%%
%%%%%%%%%%%%%%%%%%%%%%%%%%%%%%%%%%%%
\begin{tabular}{|c|c|c|}
\hline
$x_1$ & $x_2$ & $y$\\
\hline
$-2.6$ & $6.6$ & 1\\
$1.4$ & $1.6$ & 2\\
$-2.5$ & $1.2$ & 2\\
\hline
\end{tabular}
}

\newcommand{\HWStudSolBB}{
%%%%%%%%%%%%%%%%%%%%%%%%%%%%%%%%%%%%
%%
%%.   YOUR SOLUTION FOR PROBLEM 2.b) BELOW THIS COMMENT
%%
%%%%%%%%%%%%%%%%%%%%%%%%%%%%%%%%%%%%
\begin{tabular}{|c|c|c|}
\hline
$x_1$ & $x_2$ & $y$\\
\hline
$-2.6$ & $6.6$ & 1\\
$1.4$ & $1.6$ & 1\\
$-2.5$ & $1.2$ & 2\\
\hline
\end{tabular}
}

\newcommand{\HWStudSolBC}{
%%%%%%%%%%%%%%%%%%%%%%%%%%%%%%%%%%%%
%%
%%.   YOUR SOLUTION FOR PROBLEM 2.c) BELOW THIS COMMENT
%%
%%%%%%%%%%%%%%%%%%%%%%%%%%%%%%%%%%%%
Assuming uniform weights for every point, the predicted label would be the most frequent class in the dataset regardless of the specific input being evaluated. If the dataset has a uniformly distributed set of classes (exact amount of each class) then one can pick randomly. 
\vspace{2cm}
}

\newcommand{\HWStudSolBD}{
%%%%%%%%%%%%%%%%%%%%%%%%%%%%%%%%%%%%
%%
%%.   YOUR SOLUTION FOR PROBLEM 2.d) BELOW THIS COMMENT
%%
%%%%%%%%%%%%%%%%%%%%%%%%%%%%%%%%%%%%
Zero. There is no learning involved. All the heavy lifting happens during inference. 
\vspace{2cm}
}


