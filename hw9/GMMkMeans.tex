% !TEX root = exam.tex
\ifthenelse{\equal{\type}{booklet}}{
\newcommand{\GMMkMeansStudSolA}{
%%%%%%%%%%%%%%%%%%%%%%%%%%%%%%%%%%%%
%%
%%.   YOUR SOLUTION FOR PROBLEM A BELOW THIS COMMENT
%%
%%%%%%%%%%%%%%%%%%%%%%%%%%%%%%%%%%%%
\begin{align*}
\text{LL} &= \ln \prod_{x_i \in X} p\left(x_i; \theta \right)\\
&= \sum_{x_i \in X} \ln p\left(x_i; \theta \right)\\
&= \sum_{x_i \in X} \ln \sum_{k = 1}^K p\left(x_i ; \mu_k, \sigma_k \right)\\
&= \sum_{x_i \in X} 
		\ln
			\sum_{k = 1}^K
				\pi_k \mathcal{N}\left(
					x_i | \mu_k, \sigma_k
				\right)\\
&= \sum_{x_i \in X} 
		\ln
			\sum_{k = 1}^K
				\pi_k  \left(2\pi \sigma_k^2\right)^{-\frac{1}{2}} \exp \left(
					-\frac{1}{2\sigma_k^2} \left( x_i - \mu_k \right)^2
				\right)
\end{align*}
Where the last equality follows because we are dealing 1-d data so there is no need for determinant or matrix inverses.
}

\newcommand{\GMMkMeansStudSolB}{
%%%%%%%%%%%%%%%%%%%%%%%%%%%%%%%%%%%%
%%
%%.   YOUR SOLUTION FOR PROBLEM A BELOW THIS COMMENT
%%
%%%%%%%%%%%%%%%%%%%%%%%%%%%%%%%%%%%%
\begin{align*}
p\left( z_i = k | x_i; \theta^{(t)} \right) &= \frac{p\left(z_i = k, x_i; \theta^{(t)} \right)}{p\left( x_i; \theta^{(t)} \right)}\\
&= \frac{p\left(z_i = k; \theta^{(t)} \right) p\left( x_i | z_i = k; \theta^{(t)} \right)}{p\left( x_i; \theta^{(t)} \right)}\\
&= \frac{p\left(z_i = k; \theta^{(t)} \right) p\left( x_i | z_i = k; \theta^{(t)} \right)}
	{\sum_{j = 1}^K p\left( x_i, z_i = j; \theta^{(t)}_k \right)}\\
&= \frac{p\left(z_i = k; \theta^{(t)} \right) p\left( x_i | z_i = k; \theta^{(t)} \right)}
	{\sum_{j = 1}^K p\left(z_i = j; \theta^{(t)} \right)  p\left( x_i | z_i = j; \theta^{(t)} \right)}\\
&= \frac{\pi_k \mathcal{N}\left(x_i | \mu_k^{(t)}, \sigma_k^{(t)} \right)}
	{\sum_{j = 1}^K \pi_j \mathcal{N}\left(x_i | \mu_j^{(t)}, \sigma_j^{(t)} \right)}
\end{align*}
}

\newcommand{\GMMkMeansStudSolC}{
%%%%%%%%%%%%%%%%%%%%%%%%%%%%%%%%%%%%
%%
%%.   YOUR SOLUTION FOR PROBLEM A BELOW THIS COMMENT
%%
%%%%%%%%%%%%%%%%%%%%%%%%%%%%%%%%%%%%
\begin{align*}
\bE_{z_i| x_i; \theta^{(t)}}[\log p(x_i, z_i; \theta)] &= 
	\bE_{z_i| x_i; \theta^{(t)}}\left[
		\log \prod_{k=1}^K p\left(x_i, z_i; \theta_k\right)^{\delta\left(z_i = k\right)}
	\right]\\
&= \bE_{z_i| x_i; \theta^{(t)}}\left[
		\sum_{k=1}^K \log p\left(x_i, z_i = k; \theta_k\right)^{\delta\left(z_i = k\right)}
	\right]\\
&= \sum_{k=1}^K \bE_{z_i| x_i; \theta^{(t)}}\left[
		\log p\left(x_i, z_i = k; \theta_k\right)^{\delta\left(z_i = k\right)}
	\right]\\
&= \sum_{k=1}^K \bE_{z_i| x_i; \theta^{(t)}}\left[
		\delta\left(z_i = k\right)		
		\log p\left(x_i, z_i = k; \theta_k\right)
	\right]\\
&= \sum_{k=1}^K \bE_{z_i| x_i; \theta^{(t)}}\left[
		\delta\left(z_i = k\right)		
		\log p\left(z_i = k | x_i; \theta_k\right) p\left( x_i; \theta_k \right)
	\right]\\
&= \sum_{k=1}^K \bE_{z_i| x_i; \theta^{(t)}}\left[
		\delta\left(z_i = k\right)		
		\log \pi_k p\left( x_i; \theta_k \right)
	\right]\\
&= \sum_{k=1}^K \bE_{z_i| x_i; \theta^{(t)}}\left[
		\delta\left(z_i = k\right)
	\right]
	\log \pi_k p\left( x_i; \theta_k \right)\\
&= \sum_{k=1}^K p\left(z_i = k | x_i; \theta^{(t)} \right)
	\log \pi_k p\left( x_i; \theta_k \right)\\
&= \sum_{k=1}^K z_{ik}
	\log \pi_k p\left( x_i; \theta_k \right)\\
&= \sum_{k=1}^K z_{ik} \left(
		\log \pi_k + \log p\left( x_i; \theta_k \right)
	\right)\\
&= \sum_{k=1}^K z_{ik} \left(
		\log \pi_k + \log \mathcal{N}\left(x_i | \mu_k, \sigma_k \right)
	\right)
\end{align*}

The idea of using a product with an indicator function as an exponent was taken from the textbook. I really liked that trick so decided to use it. Apparently there are easier ways to derive this result by just using the definition of an expectation.
}

\newcommand{\GMMkMeansStudSolD}{
%%%%%%%%%%%%%%%%%%%%%%%%%%%%%%%%%%%%
%%
%%.   YOUR SOLUTION FOR PROBLEM A BELOW THIS COMMENT
%%
%%%%%%%%%%%%%%%%%%%%%%%%%%%%%%%%%%%%
Summarizing:
\begin{align*}
\pi_k^{(t+1)} &= \frac{1}{N} \sum_{i = 1}^N z_{ik}\\
\mu_k^{(t+1)} &= \frac{\sum_{i=1}^N z_{ik}x_i}{N\pi_k^{(t+1)}} \\
\sigma_k^{2, (t+1)} &= \frac{
	\sum_{i=1}^N z_{ik} \left(
		x_i - \mu_k^{(t+1)}
	\right)^2
}{N\pi_k^{(t+1)}}
\end{align*}

The whole derivation can be found at the end of this report. For some reason this template does not like boxes that span multiple pages.
}

\newcommand{\GMMkMeansStudSolE}{
%%%%%%%%%%%%%%%%%%%%%%%%%%%%%%%%%%%%
%%
%%.   YOUR SOLUTION FOR PROBLEM A BELOW THIS COMMENT
%%
%%%%%%%%%%%%%%%%%%%%%%%%%%%%%%%%%%%%
\begin{align*}
\pi_k^{(t+1)} &= \frac{1}{K} \hspace{0.2cm} \forall k\\
\sigma_k^{(t+1)} &= c \hspace{0.2cm} \forall k\\
c \downarrow 0
\end{align*}

Other relations include:\\
1) distance measure is different. k-Means uses Euclidean distance whereas GMM uses a Gaussian probability.\\
2) kMeans assumes the data is spherically clustered, as consequence of using Euclidean distance.
}
%% z_{ik} &\approx \delta\left(k = \argmin_j ||x_i - \mu_j||^2 \right)

\newcommand{\GMMkMeansStudSolF}{
%%%%%%%%%%%%%%%%%%%%%%%%%%%%%%%%%%%%
%%
%%.   YOUR SOLUTION FOR PROBLEM A BELOW THIS COMMENT
%%
%%%%%%%%%%%%%%%%%%%%%%%%%%%%%%%%%%%%
\vspace{12cm}
}
 %The students have to fill this file to print the solution
}{
\input{GMMkMeans_OurSolution} %This file will not be provided to students since it contains the solution
}

% Problem Explanation:
% - first argument is the number of points
% - second argument is the title and the text
\examproblem{16}{Gaussian Mixture Models \& EM\\
Consider a Gaussian mixture model with $K$ components ($k\in\{1, \ldots, K\}$), each having mean $\mu_k$, variance $\sigma_k^2$, and mixture weight $\pi_k$. All these are parameters to be learned, and we subsume them in the set $\theta$. Further, we are given a dataset $X = \{x_i\}$, where $x_i \in \mathbb{R}$. We also use $Z = \{z_{i}\}$ to denote the latent variables, such that $z_{i} = k$ implies that $x_i$ is generated from the $k^{th}$ Gaussian.
}

%%%%%%%%%%%%%%%%%%%%%%%%%%%%%%%%%%%%%%
%%%%%  BEGINNING OF SUBPROBLEMS LIST

\begin{enumerate}

 % Subproblem description
\examproblempart{What is the log-likelihood of the data $\log p(X; \theta)$ according to the Gaussian Mixture Model? (use $\mu_k$, $\sigma_k$, $\pi_k$, $K$, $x_i$, and $X$). Don't use any abbreviations. \\}

\bookletskip{0.0}   %in inches

% Solution box
 \framebox[14.7cm][l]{
 \begin{minipage}[b]{14.7cm}
 \inbooklet{Your answer: \GMMkMeansStudSolA}

 \solution{\GMMkMeansSolA}
 \end{minipage}
 }

  % Subproblem description
\examproblempart{For learning $\theta$ using the EM algorithm, we need the conditional distribution of the latent variables $Z$ given the current estimate of the parameters $\theta^{(t)}$ (we will use the superscript ($t$) for parameter estimates at step $t$). What is the posterior probability $p(z_{i} = k |x_i; \theta^{(t)})$? To simplify, wherever possible, use $\mathcal{N}(x_i | \mu_k, \sigma_k)$ to denote a Gaussian distribution over $x_i\in\mathbb{R}$ having mean $\mu_k$ and variance $\sigma_k^2$.}

\bookletskip{0.0}   %in inches

% Solution box
 \framebox[14.7cm][l]{
 \begin{minipage}[b]{14.7cm}
 \inbooklet{Your answer: \GMMkMeansStudSolB}

 \solution{\GMMkMeansSolB}
 \end{minipage}
 }

% Subproblem description
 \examproblempart{Find $\bE_{z_i| x_i; \theta^{(t)}}[\log p(x_i, z_i; \theta)]$. Denote $p(z_i = k| x_i; \theta^{(t)})$ as $z_{ik}$, and use all previous notation simplifications.}

\bookletskip{0.0}   %in inches

% Solution box
 \framebox[14.7cm][l]{
 \begin{minipage}[b]{14.7cm}
 \inbooklet{Your answer: \GMMkMeansStudSolC}

 \solution{\GMMkMeansSolC}
 \end{minipage}
 }

% Subproblem description
\examproblempart{$\theta^{(t+1)}$ is obtained as the maximizer of $\sum_{i=1}^N \bE_{z_i| x_i; \theta^{(t)}}[\log p(x_i, z_i; \theta)]$. Find $\mu_k^{(t+1)}$, $\pi_k^{(t+1)}$, and $\sigma_k^{(t+1)}$, by using your answer to the previous question.}

\bookletskip{0.0}   %in inches

% Solution box
 \framebox[14.7cm][l]{
 \begin{minipage}[b]{14.7cm}
 \inbooklet{Your answer: \GMMkMeansStudSolD}

 \solution{\GMMkMeansSolD}
 \end{minipage}
 }

 \examproblempart{How are kMeans and Gaussian Mixture Model related? (There are three conditions)}

\bookletskip{0.0}   %in inches

% Solution box
 \framebox[14.7cm][l]{
 \begin{minipage}[b]{14.7cm}
 \inbooklet{Your answer: \GMMkMeansStudSolE}

 \solution{\GMMkMeansSolE}
 \end{minipage}
 }


  %%%%%%%%%%%% END OF SUBPROBLEMS LIST

 \end{enumerate}
